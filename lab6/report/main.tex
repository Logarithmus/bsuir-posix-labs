\section{Цели работы}
\label{sec:target}

Изучить потоки в Linux.

\section{Выполнение работы}
\label{sec:job} 
\subsection{Общее задание}
\label{sec:job:general_task}

\begin{enumerate}[listparindent=\fivecharsapprox]
	\item Изучить теоретическую часть лабораторной работы.
	\item Написать программу, создающую два дочерних потока. Родительский процесс и два дочерних потока должны выводить на экран свой \lstinline{tid}, \lstinline{pid} родительского процесса и текущее время в формате: часы:минуты:секунды:миллисекунды.
\lstinputlisting[language=c]{../pthread.c}

\fastpicture{1.png}{pic1}{Результат работы \lstinline{pthread.c}}

\end{enumerate}

\subsection{Индивидуальное задание}
\label{sec:job:personal_task}

Вариант 1. Написать программу нахождения массива \(K\) последовательных значений функции \(y[i] = \sin{(2 \pi i / N)}\, (i = 0, 1, 2, ..., K - 1)\) с использованием ряда Тейлора.
Пользователь задаёт значения \(K\), \(N\) и количество \(n\) членов ряда Тейлора. Для расчета каждого члена ряда Тейлора запускается отдельный поток. Каждый поток выводит на экран свой \lstinline{tid} и рассчитанное значение ряда.
Головной процесс ожидает завершения работы всех потоков и суммирует все члены ряда Тейлора. Полученное значение \(y[i]\) он записывает в файл результата. 
\lstinputlisting[language=c]{../personal.c}

На рисунках \ref{fig:pic2} и \ref{fig:pic3} показан результат выполнения \lstinline{personal.c}.

\fastpicture{2.png}{pic2}{Результат работы \lstinline{personal.c}}

\fastpicture{3.png}{pic3}{Содержимое файла \lstinline{out.txt}}

\section{Выводы}
\label{sec:conclusion}

В ходе лабораторной работы были изучены основы работы с потоками в POSIX-совместимых системах (функции \lstinline{pthread_create}, \lstinline{pthread_join}, \lstinline{pthread_self}).
Основной источник информации при выполнении заданий -- man-страницы <<POSIX Programmer's Manual>>.

При выполнении индивидуального задания для вычисления каждого слагаемого ряда Тейлора создавался новый поток.
Создание потоков операционной системы несёт за собой накладные расходы из-за переключения контекста. По этой причине на практике создаётся пул потоков по количеству процессорных ядер.
Когда поток завершает выполнение задачи, планировщик потоков назначает ему новую задачу.
Основные подходы к решению данного вопроса -- \textit{green threads} и \textit{async / await}.
