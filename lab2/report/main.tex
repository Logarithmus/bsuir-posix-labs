\section{Цели работы}
\label{sec:target}

Изучить основные системные вызовы и функции в ОС UNIX

\section{Выполнение работы}
\label{sec:job} 
\subsection{Общее задание}
\label{sec:job:general_task}

\begin{enumerate}[listparindent=\fivecharsapprox]

	\item Изучить теоретическую часть лабораторной работы.
	\item В консольном режиме создать в домашней папке подкаталог: /номер\_группы/ФИО\_студента, где в дальнейшем будут храниться все файлы студента.
		Перейти в корневой каталог и вывести его содержимое, используя команды \lstinline{dir} и \lstinline{ls --all} \picref{fig:pic1}, проанализировать различия.
В отличие от \lstinline{dir}, команда \lstinline{ls --all} также выводит скрытые файлы и папки (имя которых начинается с точки).

\fastpicture{1.png}{pic1}{Сравнение команд \lstinline{dir} и \lstinline{ls --all}}

	\item Проверить действие команд \lstinline{ps} (рис. \ref{fig:pic2}), \lstinline{ps x} (рис. \ref{fig:pic3}), \lstinline{top} \picref{fig:pic4}, \lstinline{htop} \picref{fig:pic5}.

\fastpicture{2.png}{pic2}{Вывод команды \lstinline{ps}}
\fastpicture{3.png}{pic3}{Вывод команды \lstinline{ps x}}
\fastpicture{4.png}{pic4}{Вывод команды \lstinline{top}}
\fastpicture{5.png}{pic5}{Вывод команды \lstinline{htop}}

Найти в справочной системе, используя команду \lstinline{man}, справку по функциям \lstinline{fprintf} \picref{fig:pic6}, \lstinline{fputc} \picref{fig:pic7} и команде \lstinline{ls} \picref{fig:pic8}.

\fastpicture{6.png}{pic6}{Страница \lstinline{man} для функции \lstinline{fprintf}}
\fastpicture{7.png}{pic7}{Страница \lstinline{man} для функции \lstinline{fputc}}
\fastpicture{8.png}{pic8}{Страница \lstinline{man} для команды \lstinline{ls}}

	\item В текстовом редакторе \lstinline{joe} (вызов: \lstinline{joe 1.c}) написать программу \lstinline{1.c}, выводящую на экран фразу “HELLO SUSE Linux” \picref{fig:pic9}. Компилировать полученную программу компилятором gcc: \lstinline{gcc 1.c –o 1.o}. Запустить полученный файл \lstinline{1.o} на выполнение: \lstinline{./1.o} \picref{fig:pic10}.
		

\fastpicture{9.png}{pic9}{Файл \lstinline{1.c}, открытый в текстовом редакторе \lstinline{joe}}
\fastpicture{10.png}{pic10}{Компиляция и выполнение \lstinline{1.o}}

	\item Написать скрипт, выводящий в консоль и в файл все аргументы командной строки \picref{fig:pic11}.

\fastpicture{11.png}{pic11}{Скрипт для вывода всех аргументов командной строки}

	\item Написать скрипт, выводящий в файл (имя файла задаётся пользователем в качестве первого аргумента командной строки) имена всех файлов с заданным расширением (третий аргумент командной строки) из заданного каталога (имя каталога задаётся пользователем в качестве второго аргумента командной строки) \picref{fig:pic12}.

\fastpicture{12.png}{pic12}{Скрипт для вывода имён файлов с заданным расширением}

	\item Написать скрипт, компилирующий и запускающий программу (имя исходного файла и исполняемого файла результата задаётся пользователем в качестве аргументов командной строки).
В случае ошибок при компиляции вывести на консоль сообщение об ошибках и не запускать программу на выполнение \picref{fig:pic13}.

\fastpicture{13.png}{pic13}{Cкрипт для компиляции и запуска программы}

\end{enumerate}

\subsection{Индивидуальное задание}
\label{sec:job:personal_task}

Вариант 7. Написать скрипт, подсчитывающий суммарный размер файлов в заданном
каталоге и всех его подкаталогах (имя каталога задаётся пользователем в
качестве аргумента командной строки). Скрипт выводит результаты подсчёта
в файл (второй аргумент командной строки) в виде: каталог (полный путь),
суммарный размер файлов число просмотренных файлов.

\lstinputlisting[language=bash]{../script.sh}

На рисунке \ref{fig:pic14} показан результат выполнения скрипта.

\fastpicture{14.png}{pic14}{Результат работы скрипта для подсчёта файлов}

\section{Выводы}
\label{sec:out}

В ходе лабораторной работы были изучены основы операционной системы GNU/Linux, основные команды интерпретатора bash, а также основные утилиты из состава GNU Coreutils. Отличие \lstinline{ps x} от \lstinline{ps} заключается в том, что вторая команда выводит только подключённые к терминалу процессы. Процессы, которые выполняются не в терминале, можно посмотреть первой командой.

Был написан скрипт для подсчёта суммарного размера файлов во всех подкаталогах. Во избежание путаницы и смешивания данных, полученных в ходе разных запусков скрипта, в самом начале работы скрипта выходной файл удаляется.
