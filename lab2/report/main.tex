\section{Цели работы}
\label{sec:target}

Изучить основные системные вызовы и функции в ОС UNIX для работы с файлами и каталогами.

\section{Выполнение работы}
\label{sec:job} 
\subsection{Общее задание}
\label{sec:job:general_task}

\begin{enumerate}[listparindent=\fivecharsapprox]
	\item Изучить теоретическую часть лабораторной работы.
	\item Написать программу вывода сообщения на экран.
	\item Написать программу ввода символов с клавиатуры и записи их в файл (в качестве аргумента при запуске программы вводится имя файла).
Для чтения или записи файла использовать только функции посимвольного ввода-вывода \lstinline{getc()}, \lstinline{putc()}, \lstinline{fgetc()}, \lstinline{fputc()}.
Предусмотреть выход после ввода определённого символа(например: \lstinline{Ctrl+F}).
Предусмотреть контроль ошибок открытия/закрытия/чтения файла.
	\item Написать программу вывода содержимого текстового файла на экран (в качестве аргумента при запуске программы передаётся имя файла, второй аргумент (\textbf{N}) устанавливает вывод по группам строк (по \textbf{N} строк) или сплошным текстом (\textbf{N = 0})).
Для вывода очередной группы строк необходимо ожидать нажатия пользователем любой клавиши.
Для чтения или записи файла использовать только функции посимвольного ввода-вывода \lstinline{getc()}, \lstinline{putc()}, \lstinline{fgetc()}, \lstinline{fputc()}.
Предусмотреть контроль ошибок открытия/закрытия/чтения/записи файла.
	\item Написать программу копирования одного файла в другой.
В качестве параметров при вызове программы передаются имена первого и второго файлов.
Для чтения или записи файла использовать только функции посимвольного ввода-вывода \lstinline{getc()}, \lstinline{putc()}, \lstinline{fgetc()}, \lstinline{fputc()}.
Предусмотреть копирование прав доступа к файлу и контроль ошибок открытия/закрытия/чтения/записи файла.
	\item Написать программу вывода на экран содержимого текущего и корневого каталогов.
Предусмотреть контроль ошибок открытия/закрытия/чтения каталога.

\end{enumerate}

\subsection{Индивидуальное задание}
\label{sec:job:personal_task}

Вариант 3. Для заданного каталога (аргумент 1 командной строки) и всех его подкаталогов вывести в заданный файл (аргумент 2 командной строки) и на консоль имена файлов, их размер и дату создания, удовлетворяющих заданным условиям: 1 – размер файла находится в заданных пределах от \textbf{N1} до \textbf{N2} (\textbf{N1}, \textbf{N2} задаются в аргументах командной строки), 2 – дата создания находится в заданных пределах от \textbf{M1} до \textbf{M2} (\textbf{M1}, \textbf{M2} задаются в аргументах командной строки).
4. Найти совпадающие по содержимому файлы в двух заданных каталогах

\lstinputlisting[language=c]{../personal.c}

На рисунке \ref{fig:pic14} показан результат выполнения скрипта.

\fastpicture{14.png}{pic14}{Результат работы скрипта для подсчёта файлов}

\section{Выводы}
\label{sec:out}

В ходе лабораторной работы были изучены основы операционной системы GNU/Linux, основные команды интерпретатора bash, а также основные утилиты из состава GNU Coreutils. Отличие \lstinline{ps x} от \lstinline{ps} заключается в том, что вторая команда выводит только подключённые к терминалу процессы. Процессы, которые выполняются не в терминале, можно посмотреть первой командой.

Был написан скрипт для подсчёта суммарного размера файлов во всех подкаталогах. Во избежание путаницы и смешивания данных, полученных в ходе разных запусков скрипта, в самом начале работы скрипта выходной файл удаляется.
