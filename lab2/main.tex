\section{Цели работы}
\label{sec:target}

Изучить основные объекты, команды, типы данных и операторы управления интерпретатора \textit{bash}; создать скрипт-файл.

\section{Выполнение работы}
\label{sec:job}

\subsection{Индивидуальное задание}
\label{sec:job:task}

3. Для заданного каталога (аргумент 1 командной строки) и всех его
подкаталогов вывести в заданный файл (аргумент 2 командной строки) и на
консоль имена файлов, их размер и дату создания, удовлетворяющих заданным
условиям: 1 – размер файла находится в заданных пределах от N1 до N2 (N1,N2
задаются в аргументах командной строки), 2 – дата создания находится в
заданных пределах от M1 до M2 (M1,M2 задаются в аргументах командной
строки).

\subsection{Исходный код}
\label{sec:job:code}

\lstinputlisting[language=bash,numbers=left]{main.c}

\section{Выводы}
\label{sec:out}

В ходе лабораторной работы были изучены основы операционной системы GNU/Linux, основные команды интерпретатора bash, а также основные утилиты из состава GNU Coreutils.
Был написан скрипт для подсчёта суммарного размера файлов во всех подкаталогах.
