\section{Цели работы}
\label{sec:target}

Изучить основные системные вызовы и функции в ОС UNIX для работы с файлами и каталогами.

\section{Выполнение работы}
\label{sec:job} 
\subsection{Общее задание}
\label{sec:job:general_task}

\begin{enumerate}[listparindent=\fivecharsapprox]
	\item Изучить теоретическую часть лабораторной работы.
	\item Написать программу вывода сообщения на экран \picref{fig:pic1}.
\lstinputlisting[language=c]{../print.c}

\fastpicture{1.png}{pic1}{Результат работы \lstinline{print.c}}

	\item Написать программу ввода символов с клавиатуры и записи их в файл (в качестве аргумента при запуске программы вводится имя файла).
Для чтения или записи файла использовать только функции посимвольного ввода-вывода \lstinline{getc()}, \lstinline{putc()}, \lstinline{fgetc()}, \lstinline{fputc()}.
Предусмотреть выход после ввода определённого символа(например: \lstinline{Ctrl+F}).
Предусмотреть контроль ошибок открытия/закрытия/чтения файла \picref{fig:pic2}.
\lstinputlisting[language=c]{../write_to_file.c}

\fastpicture{2.png}{pic2}{Результат работы \lstinline{write_to_file.c}}

	\item Написать программу копирования одного файла в другой.
В качестве параметров при вызове программы передаются имена первого и второго файлов.
Для чтения или записи файла использовать только функции посимвольного ввода-вывода \lstinline{getc()}, \lstinline{putc()}, \lstinline{fgetc()}, \lstinline{fputc()}.
Предусмотреть копирование прав доступа к файлу и контроль ошибок открытия/закрытия/чтения/записи файла \picref{fig:pic3}.
\lstinputlisting[language=c]{../copy_file.c}

\fastpicture{3.png}{pic3}{Результат работы \lstinline{copy_file.c}}

	\item Написать программу вывода на экран содержимого текущего и корневого каталогов.
		Предусмотреть контроль ошибок открытия/закрытия/чтения каталога \picref{fig:pic4}.
\lstinputlisting[language=c]{../mini_ls.c}

\fastpicture{4.png}{pic4}{Результат работы \lstinline{mini_ls.c}}

	\item Написать программу, находящую в заданном каталоге и всех его подкаталогах все файлы заданного размера.
Имя каталога задаётся пользователем в качестве первого аргумента командной строки.
Диапазон (min -- max) размеров файлов задаётся пользователем в качестве второго и третьего аргументов командной строки.
Программа выводит результаты поиска в файл (четвертый аргумент командной строки) в виде: полный путь, имя файла, его размер.
На консоль выводится общее число просмотренных файлов \picref{fig:pic5}.
\lstinputlisting[language=c]{../mini_find.c}

\fastpicture{5.png}{pic5}{Результат работы \lstinline{mini_find.c}}

\end{enumerate}

\subsection{Индивидуальное задание}
\label{sec:job:personal_task}

Вариант 3. Для заданного каталога (аргумент 1 командной строки) и всех его подкаталогов вывести в заданный файл (аргумент 2 командной строки) и на консоль имена файлов, их размер и дату создания, удовлетворяющих заданным условиям: 1 – размер файла находится в заданных пределах от \textbf{N1} до \textbf{N2} (\textbf{N1}, \textbf{N2} задаются в аргументах командной строки), 2 – дата создания находится в заданных пределах от \textbf{M1} до \textbf{M2} (\textbf{M1}, \textbf{M2} задаются в аргументах командной строки).

\lstinputlisting[language=c]{../personal.c}

На рисунке \ref{fig:pic6} показан результат выполнения скрипта.

\fastpicture{6.png}{pic6}{Результат работы \lstinline{personal.c}}

\section{Выводы}
\label{sec:out}

В ходе лабораторной работы были изучены основы работы с файловой системой в POSIX-совместимых системах. Основной источник информации при выполнении заданий -- man-страницы.
Для рекурсивного обхода директории использовалась функция \lstinline{nftw}. Она является частью стандарта POSIX с 2007 года, т. е. уже более 13 лет.
Однако многие программисты продолжают использовать классическую реализацию, включающую в себя рекурсивный вызов \lstinline{opendir}.
