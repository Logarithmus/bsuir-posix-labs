\section{Цели работы}
\label{sec:target}

Изучить вопросы порождения и взаимодействия процессов в ОС LINUX.

\section{Выполнение работы}
\label{sec:job} 
\subsection{Общее задание}
\label{sec:job:general_task}

\begin{enumerate}[listparindent=\fivecharsapprox]
	\item Изучить теоретическую часть лабораторной работы.
	\item Написать программу, создающую два дочерних процесса с использованием двух вызовов \lstinline{fork()}.
Родительский и два дочерних процесса должны выводить на экран свой pid, pid родительского процесса и текущее время в формате: часы:минуты:секунды:миллисекунды. Используя вызов \lstinline{system()}, выполнить команду \lstinline{ps x} в родительском процессе.
Найти свои процессы в списке запущенных процессов \picref{fig:pic1}.
\lstinputlisting[language=c]{../fork.c}

\fastpicture{1.png}{pic1}{Результат работы \lstinline{fork.c}}

\end{enumerate}

\subsection{Индивидуальное задание}
\label{sec:job:personal_task}

Вариант 3. Написать программу поиска одинаковых по их содержимому файлов в двух каталогов, например, dir1 и dir2.
Пользователь задаёт имена dir1 и dir2.
В результате работы программы файлы, имеющиеся в dir1, сравниваются с файлами в dir2 по их содержимому.
Процедуры сравнения должны запускаться в отдельном процессе для каждой пары сравниваемых файлов.
Каждый процесс выводит на экран свой pid, имя файла, общее число просмотренных байтов и результаты сравнения.
Число одновременно работающих процессов не должно превышать N (вводится пользователем).

\lstinputlisting[language=c]{../personal.c}

На рисунке \ref{fig:pic2} показан результат выполнения программы \lstinline{personal.c}.

\fastpicture{2.png}{pic2}{Результат работы \lstinline{personal.c}}

\section{Выводы}
\label{sec:conclusion}

В ходе лабораторной работы были изучены основы работы с файловой системой в POSIX-совместимых системах, а также работа с процессами (функции \lstinline{fork} и \lstinline{wait}).
Основной источник информации при выполнении заданий -- man-страницы.
Для рекурсивного обхода директории использовалась функция \lstinline{nftw}. Она является частью стандарта POSIX с 2007 года, т. е. уже более 13 лет.
Однако многие программисты продолжают использовать классическую реализацию, включающую в себя рекурсивный вызов \lstinline{opendir}.
